\documentclass[10pt,a4paper]{article}
\usepackage{amsmath, amssymb, amsthm}
\usepackage{bbm}
\usepackage{mathrsfs}
\usepackage{amsfonts} %%% i.e. use 12pt type
\usepackage[dvipsnames,usenames]{color}
\textwidth=13.5cm %%% in the preamble; this will require
\baselineskip=17pt %%% after \begin{document}
\usepackage{graphicx,latexsym,bm,amsmath,amssymb,verbatim,multicol,lscape}
\usepackage{enumerate}
\usepackage {cancel}
% ----------------------------------------------------------------
\vfuzz2pt % Don't report over-full v-boxes if over-edge is small
\hfuzz2pt % Don't report over-full h-boxes if over-edge is small
% THEOREMS -------------------------------------------------------
\newtheorem{thm}{Theorem} [section]
\newtheorem{lem}{Lemma}[section]
\newtheorem{pro}{Proposition}[section]
\newtheorem{cor}{Corollary}[section]
\newtheorem{ass}{Assumption}[section]
\newtheorem{exm}{Example}[section]
\theoremstyle{definition}
\newtheorem{defn}{Definition}[section]
\theoremstyle{remark}
\newtheorem{rem}{Remark}[section]
\newtheorem{con}[thm]{Conjecture}
\numberwithin{equation}{section}
\DeclareMathOperator{\Tr}{Tr}
\DeclareMathOperator{\ind}{ind}
\DeclareMathOperator{\rank}{rank}
\allowdisplaybreaks
% MATH -----------------------------------------------------------
\newcommand{\norm}[1]{\left\Vert#1\right\Vert}
\newcommand{\abs}[1]{\left\vert#1\right\vert}
\newcommand{\set}[1]{\left\{#1\right\}}
\newcommand{\Real}{\mathbb R}
\newcommand{\eps}{\varepsilon}
\newcommand{\To}{\longrightarrow}
\newcommand{\BX}{\mathbf{B}(X)}
\newcommand{\A}{\mathcal{A}}
\newcommand{\SSS}{\stackrel}
\newcommand{\pdet}[1]{            \sqrt{\abs{   \det    \left(#1 #1^\dagger \right)   }}               }


\title{No Hermitian Lens Can Correct Color-Blindness\\---a Mathematical Proof}
\author{J.L.}
\date{\today}

\begin{document}

\maketitle

\begin{abstract}
This paper investigates the fundamental limitations of using optical lenses to correct color-blindness. We approach this problem by modelling the vision system, which includes the lens and the human eye, as a mixed quantum system. Through a mathematical proof, we demonstrate that color blindness, which can be represented as a rank deficiency problem in the ``color perception matrix $V$'', cannot be rectified by any unitary transformation corresponding to an optical lens. Our findings have implications for the design of corrective devices and challenge common practices in the field of vision science.
\end{abstract}

\section{Introduction}
Color-blindness is a well-known visual impairment where individuals have difficulty distinguishing between certain colors. Traditionally, corrective lenses have been developed to enhance contrast and help individuals perceive colors differently. However, these lenses do not truly ``correct" color-blindness in the sense of distinguishing previously-undistinguishable colors. 

In this paper, we explore the theoretical underpinnings of why no optical lens can correct color-blindness at all. We model the vision system, which includes the lens and the human eye, as a mixed quantum system, and analyze its behavior using linear algebra theory. We demonstrate that since an optical lens must be represented as a unitary transformation, it cannot reverse the rank of the color perception matrix associated with color-blindness.

\section{Background}
\subsection{Optical Lenses as Quantum Devices}
Optical lenses, while typically described in classical terms, are fundamentally quantum devices. Specifically, the evolution of light as it passes through an optical system can be described by a unitary transformation, denoted by \( U \). Unitary transformations satisfy  \( U^{-1}= U^\dagger \), corresponding to the physical principle of the Schr\"{o}dinger's equation.

\subsection{Modelling Color Perception}
Color perception in the human eye is mediated by three types of photoreceptors (cones), which respond to different wavelengths of light. These responses can be modeled as vectors in a color space. Under normal vision, the span of these receptor responses allows for the perception of a wide range of colors. However, in individuals with color-blindness, one or more of these photoreceptors is deficient, leading to a reduction in the effective rank of the color perception matrix, $V$, defined in the next section.

\section{Theoretical Analysis}
\subsection{Unitary Transformation as Quantum Process}
The propagation of light through an optical system, as a quantum process, is governed by the Schr\"{o}dinger's equation, given by:
\[
i\hbar \frac{\partial \psi}{\partial t} = H \psi
\]
where \( \psi \) is the light, \( \hbar \) is the reduced Planck's constant, and \( H \) is a Hamiltonian operator representing the lens. The solution to the Schr\"{o}dinger equation gives us the evolution $U$ of the light wave.
\begin{defn}
\begin{eqnarray}
U:\Psi \rightarrow \Psi\\
U = e^{-iHt/\hbar}
\end{eqnarray}
where $\Psi$ is the Hilbert space\cite{linearAlgebra} for light. 
\end{defn}

The relation between the light before and after passing the lens, $\psi_0,\psi_1 \in \Psi$, is given by  
\begin{eqnarray}
\psi_1 = U \psi_0 
\end{eqnarray}
\begin{lem}
$U$ is unitary if and only if $H$ is Hermitian (which it is in the context of this paper). 
\end{lem}

\subsection{Linear Transformation as Human Vision}
The vision system is a process that converts light to intensity of response of different kinds of photoreceptors. If a vision system has $n$ kinds of functioning photoreceptors, the vision system is said to be normal. 
\begin{defn}
\begin{equation}
V:\Psi \rightarrow \Real^n
\end{equation}
\end{defn}

\begin{ass}
We reasonably assume $V$ is a linear process.
\begin{eqnarray}
V(\psi_1+\psi_2)&=&V(\psi_1)+V(\psi_2)\\
V(\alpha \psi)&=&\alpha V(\psi)
\end{eqnarray}
\end{ass}
Under this assumption, we can view $V$ as a color perception matrix, and we would omit the parenthesis, notationally.


\subsection{Inability to Correct Color-Blindness}
Since color-blindness corresponds to a rank deficiency in the color perception matrix $V$, and since unitary transformations cannot increase the rank of a matrix, it follows that no optical lens (modeled as a unitary transformation) can correct color-blindness. The mathematical proof of this result is outlined below.

A normal vision system perceives color as $\vec{c}$ with $\dim(\vec{c})=n$, in the following process\footnote{For example, with proper normalization, $\vec{c}:= (r,g,b)$; or, $\vec{c}:= (r,g,b,v)$ if we account for all photoreceptors, both rods and cones.}, 
\begin{equation}
\vec{c} = V \psi 
\end{equation}
as such, a normal color vision is characterized by 
\begin{equation}
\rank(V)=n
\end{equation}


\begin{defn}
A *anopia-color-blind vision is characterized by 
\begin{equation}
\rank(V)\leq n-1
\end{equation}
or equivalently
\begin{equation}
\pdet{V}= 0
\end{equation}
and *anomaly-color-blind vision is likewise characterized by 
\begin{equation}
\pdet{V}\approx 0
\end{equation}
This quantity is analogous to the  $n$-dimensional volume of the perceived vision-space.
\end{defn}

\begin{lem}
With an empirically obtained value $\eps$, both *anopia-color-blind and *anomaly-color-blind vision are characterized by 
\begin{equation}
\pdet{V} < \eps
\end{equation}
\end{lem}


With a correction lens, where the light first passes through the lens, then into the eye, the process becomes
\begin{equation}
\vec{c}~' = V U \psi 
\end{equation}

By defining, 
\begin{equation}
V':=VU
\end{equation}
we can treat the lens and the eye as one mixed vision system, $\vec{c}~' = V' \psi $.

\begin{lem}
\begin{equation}
V' V'^\dagger = V V^\dagger 
\end{equation}
\end{lem}
\textbf{Proof:} 
\begin{eqnarray}
V' V'^\dagger = VU (VU)^\dagger 
                      =  V \cancel{ U U^\dagger } V^\dagger 
                      =  V V^\dagger 
\end{eqnarray}

\begin{thm}
\begin{equation}
\pdet{V} < \eps \iff  \pdet{V'} < \eps
\end{equation}
Hence, for any Hermitian lens used to correct the vision, a color-blind vision will remain color blind, without the cost of other sense. 
\end{thm}

\section{Discussion}

The implications of the results presented in this paper are profound for both the scientific understanding of color vision and the design of corrective devices for color-blindness. Our theoretical analysis shows that color-blindness, which is fundamentally a rank-deficiency problem in the color perception matrix, cannot be corrected by any unitary transformation that an optical lens might perform.

\subsection{Limitations of Optical Corrections}
The results suggest that the limitations of optical lenses are inherent. Since lenses function by modifying the incoming light through linear transformations (which are unitary in nature), they are constrained by the mathematical properties of these transformations. While certain optical lenses may improve contrast or shift the perceived color spectrum, they do not alter the fundamental issue of color receptor deficiency in the eye. This finding aligns with the fact that current color-blindness correction lenses often do not restore normal color vision but instead enhance specific visual cues.

\subsection{Broader Implications for Vision Science}
The linearity of perception and the rank-preserving nature of unitary transformations prompt a reconsideration of how we model human vision and the correction of its deficiencies. Color-blindness represents a fundamental problem in the mapping from the space of light inputs to the space of perceived colors, which cannot be resolved by any unitary process alone. This highlights the need for alternative approaches that go beyond traditional optical methods, potentially involving non-linear transformations or neural-level interventions.

Moreover, while this paper focuses on color-blindness, the broader framework can be applied to other visual impairments that involve rank deficiencies or other forms of degeneracy in the perceptual system. Understanding the mathematical and physical constraints of optical systems may provide insights into the fundamental limits of sensory correction and enhancement technologies.


\section*{Conflicts of Interest}
The authors declare no conflict of interest. The author is not a shareholder of any color correcting lenses companies, including but not limited to: enchroma, colormax, pilestone, colorlite, etc. 



\section{Conclusion}
This paper demonstrates that color-blindness cannot be corrected by any Hermitian optical lens, as such lenses are represented by unitary transformations that preserve the rank of the color perception matrix. Consider the extra cognitive burden introduced by the lenses, they would only worsen the condition. Our findings suggest that alternative approaches, potentially involving non-optical methods, are necessary for addressing color-blindness. Future work may explore the implications of this result for the design of visual aids and for our understanding of the limits of optical correction.

\bibliographystyle{plain}
\bibliography{references}

\end{document}





